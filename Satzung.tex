\documentclass[10pt,a4paper,oneside,parskip=half]{scrartcl}
\usepackage[utf8]{inputenc}
\usepackage[ngerman]{babel}
\usepackage[T1]{fontenc}
\usepackage[margin=2.5cm]{geometry}
\usepackage{lmodern}
\usepackage{graphicx}
\usepackage{tabularx}
\usepackage{xfrac}
\usepackage[juratotoc]{scrjura}
\usepackage{lastpage}
\usepackage{scrpage2}
\usepackage{hyperref}

\title{Satzung}
\date{}
\author{Ring Politischer Jugend Oldenburg}

\pagestyle{scrheadings}
\renewcommand*{\titlepagestyle}{scrheadings}

\begin{document}
\cfoot{Seite \thepage\ von \pageref{LastPage}}
\maketitle

\begin{contract}

\Clause{title={Name, Sitz und Geschäftsjahr des Vereins}}
Der Verein heißt Ring Politischer Jugend Oldenburg.

Der Verein soll in das Vereinsregister eingetragen werden und trägt dann den Zusatz „e.\,V.“

Der Sitz des Vereins ist die Stadt Oldenburg.

Geschäftsjahr ist das Kalenderjahr.

\Clause{title={Vereinszweck}}
Der Verein dient dem Zweck, die politische Bildung junger Menschen zu fördern und antidemokratischen Einflüssen entgegenzutreten.

Die Ziele des Vereins sollen durch Bildungs- und Informationsveranstaltungen, Seminaren und Kampagnen erreicht werden.

Der Verein verfolgt ausschließlich und unmittelbar gemeinnützige und mildtätige Zwecke im Sinne des Abschnitts "`Steuerbegünstigte Zwecke"' der Abgabenordnung. Der Verein ist selbstlos tätig und verfolgt nicht in erster Linie eigenwirtschaftliche Zwecke. Mittel des Vereins dürfen nur für die satzungsmäßigen Zwecke verwendet werden. Die Mitglieder erhalten keine Zuwendungen aus Mitteln des Vereins. Es darf keine Person durch Ausgaben, die dem Zweck der Körperschaft fremd sind,
oder durch unverhältnismäßig hohe Vergütungen begünstigt werden.

\Clause{title={Mitgliedschaft}}
Mitglied des Vereins können nur die Jugendorganisationen politischer Parteien werden, die im Rat der Stadt Oldenburg vertreten sind. Ausnahmen können die bestehenden Mitglieder einstimmig beschließen.

Über die Aufnahme beschließt die Mitgliederversammlung.

Die Mitgliedschaft wird beendet
\begin{enumerate}
\item durch Auflösung der Jugendorganisation
\item durch Austritt
\item durch Ausschluss
%TODO Inaktivitätsregelung?
\end{enumerate}

\Clause{title={Organe}}
Organe des Vereins sind
\begin{enumerate}
\item die Mitgliederversammlung;
\item der Vorstand, bestehend aus dem Vorsitzenden, seinem Stellvertreter und dem Schatzmeister sowie bis zu vier Beisitzern; der Vorstand wird von der Mitgliederversammlung für die Dauer von zwei Jahren gewählt; Wiederwahl ist zulässig;
\item der Beirat, der auf Beschluss des Vorstands aus geeignet erscheinenden, hierfür ehrenamtlich tätigen Personen gebildet werden kann. %TODO Beisitzer + Beirat? Weg?
\end{enumerate}

\Clause{title={Mitgliederversammlung}}
Die ordentliche Mitgliederversammlung ist quartalsweise abzuhalten. Sie beschließt insbesondere über %TODO jährlich?
\begin{enumerate}
\item die Wahl von Vorstandsmitgliedern,
\item die Verwendung und Verteilung der zur Verfügung gestellten Mittel, %TODO Handlungsfähigkeit des Vorstands!
\item die Ausschließung eines Mitgliedes,
\item die Auflösung des Ring Politische Jugend.
\end{enumerate}

Jedes Mitglied kann bis zu zwei stimmberechtigte Delegierte benennen.

Die Beschlussfähigkeit der Mitgliederversammlung ist bei Anwesenheit von mindestens der Hälfte der Mitglieder erreicht. Wenn die Beschlussfähigkeit beim ersten Termin der Vollversammlung nicht erreicht worden ist, sind beim zweiten Termin, zu dem ordnungsgemäß zu laden ist, die anwesenden Mitglieder voll beschlussfähig.

Beschlüsse erfolgen mit einfacher Mehrheit.

Der Vorstand beruft die Mitgliederversammlung in Textform ein. Die Einladung an deren letzte dem Vorstand bekannte Kontaktadresse muss mindestens zwei Wochen vor der Versammlung erfolgen.

Jedes Mitglied kann die Tagesordnung bis spätestens eine Woche vor der Mitgliedersammlung durch Nachricht an den Vorstand ergänzen.

Eine außerordentliche Mitgliederversammlung ist binnen 4 Wochen zu berufen, wenn mindestens zwei Mitglieder dies schriftlich gegenüber dem Vorstand verlangen. %TODO wenn MVs schon quartalsweise erfolgen sollen, ist das eigentlich überflüssig

\Clause{title={Vorstand des Vereins}}
Die Wahl der Vorstandsmitglieder erfolgt einzeln. Bei vorzeitigem Ausscheiden eines Vorstandsmitgliedes kann bis zur nächsten Mitgliedervollversammlung vom Vorstand ein Nachfolger bestellt werden.

Der Vorstand führt die Geschäfte des Vereins. 

Der Vorstand entscheidet in Vorstandssitzungen, zu denen er mindestens viermal jährlich, spätestens jedoch drei Wochen vor einer Mitgliederversammlung zusammentritt. Beschlüsse müssen einstimmig gefällt werden.\\
\\
Über die Sitzung wird ein Beschlussprotokoll geführt. Diese Niederschrift muss den Vorstandsmitgliedern innerhalb von zwei Wochen zugänglich gemacht werden; Einwendungen können nur innerhalb eines Monats, nachdem die Niederschrift zugänglich gemacht worden ist, erhoben werden. %TODO Sollte das Organ selbst regeln

Die Einladung ergeht über elektronische Kommunikationsmittel mit einer Frist von 10 Tagen durch den Vorsitzenden, im Falle seiner Verhinderung durch den stellvertretenden Vorsitzenden. %TODO Die letzten beiden Absätze behandeln Vorstandssitzungen, nicht die MV. Weg?

\Clause{title={Auflösung und Satzungsänderungen}}
Die Auflösung des Vereins erfordert eine Mehrheit von drei Vierteln der Mitglieder.

Satzungsänderungen müssen einstimmig der Mitgliedervollversammlung beschlossen werden. %TODO Nicht sinnvoll, Auflösung mit 75%, Anpassung mit 100%. Standardmehrheit

\Clause{title={Ausschluss}}
Der Ausschluss eines Mitglieds erfordert einen einstimmigen Beschluss der sonstigen Mitglieder, sowie einen hinreichenden Grund wie die Gefährdung des Vereinszwecks oder das Verfolgen von Zielen, die dem Vereinszweck entgegenstehen.

\end{contract}
\vspace{1cm}
Stadt Oldenburg, \today
\end{document}