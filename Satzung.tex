\documentclass[10pt,a4paper,oneside,parskip=half]{scrartcl}
\usepackage[utf8]{inputenc}
\usepackage[ngerman]{babel}
\usepackage[T1]{fontenc}
\usepackage[margin=2.5cm]{geometry}
\usepackage{lmodern}
\usepackage{graphicx}
\usepackage{tabularx}
\usepackage{xfrac}
\usepackage[juratotoc]{scrjura}
\usepackage{lastpage}
\usepackage{scrpage2}
\usepackage{hyperref}
\usepackage{tabu}

\title{Satzung}
\date{\today}
\author{Ring Politischer Jugend Oldenburg}

\pagestyle{scrheadings}
\renewcommand*{\titlepagestyle}{scrheadings}

\begin{document}
\cfoot{Seite \thepage\ von \pageref{LastPage}}
\maketitle

\begin{contract}

\Clause{title={Vereinszweck}}
Der Ring Politischer Jugend Oldenburg (RPJ) ist eine Arbeitsgemeinschaft von Mitgliedern der JUNGEN UNION IN DER STADT OLDENBURG, der JUNGSOZIALISTEN IN DER SPD OLDENBURG/AMMERLAND, der JUNGLIBERALEN OLDENBURG/AMMERLAND, der LINKSJUGEND ['solid] OLDENBURG/AMMERLAND und der GRÜNEN JUGEND OLDENBURG.

Er dient dem Zweck, die politische Bildung junger Menschen zu fördern und antidemokratischen Einflüssen entgegenzutreten.

Diese Zwecke verfolgt der Ring auf ausschließlich und unmittelbar gemeinnützige Weise im Sinne des 3.~Abschnitts der Abgabenordnung (Steuerbegünstigte Zwecke, §§ 51 ff AO). Der Ring ist selbstlos tätig und verfolgt nicht in erster Linie eigenwirtschaftliche Zwecke.

Die Ziele des Vereins sollen durch Bildungs- und Informationsveranstaltungen, Seminaren und Kampagnen erreicht werden. Dabei handelt es sich nicht um Maßnahmen der reinen Interessensvertretung einzelner Mitglieder.

\Clause{title={Name, Sitz und Geschäftsjahr des Vereins}}
Der Sitz des Ring Politischer Jugend ist Oldenburg. Geschäftsjahr ist das Kalenderjahr.

\Clause{title={Mitgliedschaft}}
Mitglied des Vereins sollen nur die Jugendorganisationen politischer Parteien werden. Ausnahmen sind möglich.

Die Mitgliedschaft wird beendet durch
\begin{enumerate}
\item Ausschluss,
\item Austritt,
\item Auflösung des Mitglieds.
\end{enumerate}

\Clause{title={Finanzierung}}
Sofern mit Tätigkeiten des Ring Politischer Jugend Oldenburg Aufwendungen verbunden sind, die einer Finanzierung bedürfen, geben die jeweiligen Mitgliedsverbände vor Entstehen der Aufwendung bekannt, inwiefern Sie Mittel für die Finanzierung der Aufwendungen bereitstellen. Die Mitgliederversammlung entscheidet im Anschluss daran über die entsprechende Tätigkeit und die Tätigung der Aufwendungen.

\Clause{title={Organe}}
Organe des Vereins sind:
\begin{enumerate}
\item die Mitgliederversammlung;
\item die Sprechenden;
\end{enumerate}

\Clause{title={Mitgliederversammlung}}
Die ordentliche Mitgliederversammlung ist quartalsweise abzuhalten. Sie beschließt insbesondere über:
\begin{enumerate}
\item die Wahl von Sprechenden,
\item die Verwendung und Verteilung der zur Verfügung gestellten Mittel,
\item die Aufnahme und den Ausschluss eines Mitgliedes,
\item die Auflösung des Ring Politischer Jugend.
\end{enumerate}

Jedes Mitglied kann bis zu zwei stimmberechtige Delegierte zur Mitgliederversammlung entsenden. Eine Benennung der Delegierten erfolgt spätestens zu Beginn der Sitzung. Eine Änderung kann auch während der Sitzung erfolgen.

Die Beschlussfähigkeit der Mitgliederversammlung ist bei Anwesenheit von mindestens einem/r Delegierten pro Mitglied erreicht.

Wenn die Beschlussfähigkeit bei einem Termin der Mitgliederversammlung nicht erreicht worden ist, ist beim folgenden Termin, zu dem ordnungsgemäß zu laden ist, die Mitgliederversammlung voll beschlussfähig.

Zur Beschlussfähigkeit muss ein ausdrücklicher schriftlicher Hinweis auf beiden Einladungen zur Mitgliederversammlung enthalten sein.

Die Sprechenden berufen die Mitgliederversammlung durch Einladung der Mitglieder per E-Mail und unter Angabe der Tagesordnung ein. Die Einladung an deren letzte den Sprechenden bekannte Kontaktadresse muss mindestens zwei Wochen vor der Versammlung erfolgen.

Die Sprechenden bestimmen die Tagesordnung. Jedes Mitglied kann eine Ergänzung der Tagesordnung bis Sitzungsbeginn beantragen.

Bei der Beschlussfassung entscheidet der Ring Politischer Jugend einstimmig. Stimmenthaltungen gelten als nicht abgegebene Stimmen.

Wahlen erfolgen offen oder auf Antrag geheim.

Alle Beschlüsse sowie die Anwesenheit müssen im Protokoll festgehalten werden.

Über die Verhandlungen der Mitgliederversammlung ist ein Protokoll zu fertigen. Das Protokoll soll den Mitgliedern innerhalb von zwei Wochen zugänglich gemacht werden. Einwände können innerhalb eines Monats nach Versand des Protokolls erhoben werden.

Eine außerordentliche Mitgliederversammlung ist unverzüglich einzuberufen, wenn mindestens zwei Mitgliedsverbände dies in Textform gegenüber den Sprechenden verlangen.

Der Ring Politischer Jugend ist offen für Gäste, die ohne Stimmberechtigung, jedoch mit Rederecht an der Versammlung teilnehmen.

\Clause{title={Sprechende}}
Die Sprechenden müssen unterschiedlichen Mitgliedern angehören. Sie müssen nicht zwingend Delegierte sein.

Die Sprechenden sollen durch alle Mitglieder rotieren. Die Rotation erfolgt in folgender Reihenfolge:
\begin{tabu} to \textwidth {l X X}
& \textbf{1. Sprecher/in} & \textbf{2. Sprecher/in} \\
1. & Grüne Jugend & Junge Union \\
2. & Junge Union & ['solid] \\
3. & ['solid] & Junge Liberale \\
4. & Junge Liberale & Jusos \\
5. & Jusos & Grüne Jugend \\
\end{tabu}

Die Sprechenden werden von den Mitgliedsverbänden benannt.

Die Amtszeit beträgt ein Jahr.

Bei vorzeitigem Ausscheiden eines/einer Sprechenden kann dessen/deren Mitgliedsverband eine/n Nachfolger/in bis zur nächsten Mitgliederversammlung benennen. Jedes Mitglied kann ein Veto einlegen. In diesem Fall ist eine außerordentliche Mitgliederversammlung einzuberufen.

Die Sprechenden führen die Geschäfte des Ring Politischer Jugend.

Die Sprechenden äußern sich politisch nur im Einvernehmen mit allen Mitgliedsverbänden.

\Clause{title={Auflösung und Satzungsänderungen}}
Die Auflösung des Ring Politischer Jugend erfolgt durch Austritt aller Mitglieder.

Satzungsänderungen müssen einstimmig von der Mitgliederversammlung beschlossen werden.

\Clause{title={Ausschluss}}
Der Ring Politischer Jugend kann ein einzelnes Mitglied mit einstimmigem Beschluss der sonstigen Mitglieder ausschließen, wenn das Mitglied die Ziele des Ring Politischer Jugend gefährdet oder gegenteilige Zwecke verfolgt.

Eine Neuaufnahme ist zulässig.

\end{contract}
\vspace{1cm}
Stadt Oldenburg, \today
\end{document}