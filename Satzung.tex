\documentclass[10pt,a4paper,oneside,parskip=half]{scrartcl}
\usepackage[utf8]{inputenc}
\usepackage[ngerman]{babel}
\usepackage[T1]{fontenc}
\usepackage[margin=2.5cm]{geometry}
\usepackage{lmodern}
\usepackage{graphicx}
\usepackage{tabularx}
\usepackage{xfrac}
\usepackage[juratotoc]{scrjura}
\usepackage{lastpage}
\usepackage{scrpage2}
\usepackage{hyperref}

\title{Satzung}
\date{}
\author{Ring Politischer Jugend Oldenburg~i.\,G.}

\pagestyle{scrheadings}
\renewcommand*{\titlepagestyle}{scrheadings}

\begin{document}
\cfoot{Seite \thepage\ von \pageref{LastPage}}
\maketitle

\begin{contract}

\Clause{title={Vereinszweck}}
Der Ring Politische Jugend Oldenburg (RPJ) ist eine Arbeitsgemeinschaft von Mitgliedern der JUNGEN UNION IN DER STADT OLDENBURG, der JUNGSOZIALISTEN IN DER SPD OLDENBURG/AMMERLAND, der JUNGLIBERALEN OLDENBURG/AMMERLAND, der LINKSJUGEND ['solid] und der GRÜNEN JUGEND OLDENBURG.

Er dient dem Zweck, die politische Bildung junger Menschen zu fördern und antidemokratischen Einflüssen entgegenzutreten.

Diese Zwecke verfolgt der Ring auf ausschließlich und unmittelbar gemeinnützige Weise im Sinne des 3. Abschnitts der Abgabenordnung (Steuerbegünstigte Zwecke", §§ 51 ff AO). Der Ring ist selbstlos tätig und verfolgt nicht in erster Linie eigenwirtschaftliche Zwecke.

Die Ziele des Vereins sollen durch Bildungs- und Informationsveranstaltungen, Seminaren und Kampagnen erreicht werden. Dabei handelt es sich nicht um Maßnahmen der reinen Interessensvertretung der Mitglieder, sondern sind der Allgemeinheit zugänglich.

\Clause{title={Name, Sitz und Geschäftsjahr des Vereins}}
Der Sitz des Ring Politische Jugend ist Oldenburg. Geschäftsjahr ist das Kalenderjahr.

\Clause{title={Mitgliedschaft}}
Mitglied des RPJ können nur diejenigen natürlichen Personen werden, welche von den jeweiligen Jugendverbänden dazu bestimmt und dem Vorstand des Vereins genannt wurden.\\
\\
Mitglieder des RPJ können nur Mitglieder solcher politischen Jugendorganisationen werden, deren Mutterparteien im Rat der Stadt Oldenburg vertreten sind. Ausnahmen hiervon sind auf Antrag und mit einstimmigen Einverständnis der sonstigen Mitglieder möglich.\\
\\
Vorausgesetzt ist weiter lediglich eine an den Vorstand gerichtete Anmeldung.\\
\\ 
Jeder Mitgliedsverband kann bis zu zwei stimmberechtigte Mitglieder benennen.\\
\\
Der Ring Politische Jugend ist offen für Gäste, die ohne Stimmberechtigung, jedoch mit Rederecht an der Versammlung teilnehmen.

Die Mitgliedschaft wird beendet
\begin{enumerate}
\item durch Tod
\item durch Austritt oder Ausschluss aus der jeweiligen Jugendorganisation
\item durch Ausschluss aus dem Ring Politische Jugend
\item durch Austritt
\end{enumerate}

\Clause{title={Finanzierung}}
Sofern mit Tätigkeiten des Ring Politische Jugend Oldenburg Aufwendungen verbunden sind, die einer Finanzierung bedürfen, geben die jeweiligen Mitgliedsverbände vor Entstehen der Aufwendung bekannt, inwiefern Sie Mittel für die Finanzierung der Aufwendungen bereitstellen. Die Mitgliederversammlung entscheidet im Anschluss daran über die entsprechende Tätigkeit und die Tätigung der Aufwendungen. Zwischen den Mitgliederversammlungen entscheidet der Vorstand.

\Clause{title={Organe}}
Organe des Vereins sind:
\begin{enumerate}
\item die Mitgliederversammlung;
\item der Vorstand, bestehend aus dem Vorsitzenden, seinem Stellvertreter und dem Schatzmeister sowie bis zu vier Beisitzern; der Vorstand wird von der Mitgliederversammlung für die Dauer von zwei Jahren gewählt; Wiederwahl ist zulässig;
\item der Beirat, der auf Beschluss des Vorstands aus geeignet erscheinenden, hierfür ehrenamtlich tätigen Personen gebildet werden kann.
\end{enumerate}

\Clause{title={Mitgliederversammlung}}
Die ordentliche Mitgliederversammlung ist quartalsweise abzuhalten. Sie beschließt insbesondere über:
\begin{enumerate}
\item die Wahl von Vorstandsmitgliedern,
\item die Verwendung und Verteilung der zur Verfügung gestellten Mittel
\item die Ausschließung eines Mitgliedes,
\item die Auflösung des Ring Politische Jugend.
\end{enumerate}

Die Beschlussfähigkeit der Mitgliederversammlung ist bei Anwesenheit von mindestens der Hälfte der Mitglieder erreicht. Wenn die Beschlussfähigkeit beim ersten Termin der Vollversammlung nicht erreicht worden ist, sind beim zweiten Termin, zu dem ordnungsgemäß zu laden ist, die anwesenden Mitglieder voll beschlussfähig.\\
\\
Zur Beschlussfähigkeit muss ein ausdrücklicher schriftlicher Hinweis auf beiden Einladungen zur Mitgliederversammlung enthalten sein.

Der Vorstand beruft die Mitgliederversammlung durch besondere Einladung der Mitglieder unter Zuhilfenahme technischer Kommunikationsmittel und unter Angabe der Tagesordnung ein. Die Einladung an deren letzte dem Vorstand bekannte Kontaktadresse muss mindestens zwei Wochen vor der Versammlung erfolgen.\\
\\
Der Vorstand bestimmt die Tagesordnung. Jedes Mitglied kann seine Ergänzung bis spätestens eine Woche vor der Verhandlung beantragen.

In der Mitgliederversammlung ist die Vertretung durch einen Stellvertreter auch bei der Ausübung des Stimmrechts zulässig, wenn dieser dem Vorstand bis Sitzungsbeginn namentlich benannt wurde. Bei der Beschlussfassung entscheidet der Ring Politischer Jugend einstimmig. Stimmenthaltungen gelten als nicht abgegebene Stimmen. Wahlen erfolgen offen oder auf Antrag von einem Mitglied geheim. Alle Beschlüsse müssen im Protokoll festgehalten werden.

Über die Verhandlungen der Mitgliederversammlung ist eine Niederschrift zu fertigen. Diese Niederschrift muss den Mitgliedern innerhalb von zwei Wochen zugänglich gemacht werden; Einwendungen können nur innerhalb eines Monats, nachdem die Niederschrift zugänglich gemacht worden ist, erhoben werden.

Eine außerordentliche Mitgliederversammlung ist binnen 4 Wochen zu berufen, wenn das Interesse des Ring Politische Jugend dies erfordert oder wenn mindestens zwei Mitgliedsverbände dies schriftlich gegenüber dem Vorstand verlangen.

\Clause{title={Vorstand des Vereins}}
Die Wahl der Vorstandsmitglieder erfolgt einzeln. Bei vorzeitigem Ausscheiden eines Vorstandsmitgliedes kann bis zur nächsten Mitgliedervollversammlung vom Vorstand ein Nachfolger bestellt werden.

Der Vorstand führt die Geschäfte des Ring Politische Jugend. 

Der Vorstand entscheidet in Vorstandssitzungen, zu denen er mindestens viermal jährlich, spätestens jedoch drei Wochen vor einer Mitgliederversammlung zusammentritt. Beschlüsse müssen einstimmig gefällt werden.\\
\\
Über die Sitzung wird ein Beschlussprotokoll geführt. Diese Niederschrift muss den Vorstandsmitgliedern innerhalb von zwei Wochen zugänglich gemacht werden; Einwendungen können nur innerhalb eines Monats, nachdem die Niederschrift zugänglich gemacht worden ist, erhoben werden.

Die Einladung ergeht über elektronische Kommunikationsmittel mit einer Frist von 10 Tagen durch den Vorsitzenden, im Falle seiner Verhinderung durch den stellvertretenden Vorsitzenden.

\Clause{title={Auflösung und Satzungsänderungen}}
Die Auflösung des Ring Politischer Jugend kann nur die Mitgliederversammlung mit einer Mehrheit von drei Vierteln der Mitglieder beschließen.

Satzungsänderungen müssen einstimmig der Mitgliedervollversammlung beschlossen werden.

\Clause{title={Ausschluss}}
Der Ring Politische Jugend kann ein einzelnes Mitglied oder einen der Mitgliedsverbände mit einstimmigem Beschluss der sonstigen Mitglieder ausschließen, wenn das Mitglied oder der Mitgliedsverband die Ziele des Ring Politischer Jugend gefährdet oder gegenteilige Zwecke verfolgt.

Ein ausgeschlossenes Mitglied oder ein ausgeschlossener Mitgliedsverband kann mit gleicher Mehrheit erneut aufgenommen werden.

\end{contract}
\vspace{1cm}
Stadt Oldenburg, \today
\end{document}